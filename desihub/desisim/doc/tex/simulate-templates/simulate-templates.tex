\documentclass[12pt]{article}
\usepackage{epsfig, mdwlist, amssymb}
\usepackage{fullpage}
%\usepackage{color}
\usepackage{url}
\usepackage{natbib}
%%\usepackage{emulateapj}
%\usepackage{footnote}
\usepackage{floatrow}
\floatsetup[table]{capposition=top}
\usepackage{deluxetable}

\newcommand{\ha}{H\ensuremath{\alpha}}
\newcommand{\hb}{H\ensuremath{\beta}}
\newcommand{\heli}{He~{\sc i}~\ensuremath{\lambda4471}}
\newcommand{\hei}{He~{\sc i}}
\newcommand{\hii}{H~{\sc ii}}

\newcommand{\oii}{[O~{\sc ii}]}
\newcommand{\oiii}{[O~{\sc iii}]}
\newcommand{\nii}{[N~{\sc ii}]}
\newcommand{\sii}{[N~{\sc ii}]}
\newcommand{\simt}{{\tt simulate-templates}}
\newcommand{\ewoii}{EW([O~{\sc ii}])}

\newcommand{\niilam}{[N~{\sc ii}]~\ensuremath{\lambda\lambda6548,84}}
\newcommand{\siilam}{[S~{\sc ii}]~\ensuremath{\lambda\lambda6716,32}}
\newcommand{\oiilam}{[O~{\sc ii}]~\ensuremath{\lambda\lambda3726,29}}
\newcommand{\oiiilam}{[O~{\sc iii}]~\ensuremath{\lambda\lambda4959,5007}}
\newcommand{\neonlam}{[Ne~{\sc iii}]~\ensuremath{\lambda3869}}

\newcommand{\ewoiilam}{EW([O~{\sc ii}]~\ensuremath{\lambda\lambda3726,29})}

\newcommand{\kms}{km~s$^{-1}$}
\newcommand{\isedfit}{{\tt iSEDfit}}

\newcommand{\apj}{ApJ}
\newcommand{\pasp}{PASP}
\newcommand{\araa}{ARA\&A}
\newcommand{\apjs}{ApJS}
\newcommand{\TODO}[1]{{\it \color{red} (#1)}}

\begin{document}
\title{Simulating State-of-the-Art Spectral Templates \\ for DESI} 

%{\bf TODO: Write a test script which ensures everything has been properly
%  installed/configured.}

\author{John Moustakas (Siena College)}
\maketitle

%\abstract{ This technote documents the construction of the preliminary
%  (currently v1.2) emission-line galaxy (ELG) templates for DESI.  }

\section{Introduction}

This document describes the methodology and code which was developed to generate
one-dimensional, noiseless, high-resolution spectral templates for DESI.  The
initial focus of this effort has been on emission-line galaxies (ELG), luminous
red galaxies (LRG), (normal) stars (STAR), including F-type standard stars
(FSTD), and white dwarfs (WD), but eventually bright-galaxy survey galaxies
(BGS), quasars (QSO), including Ly$\alpha$-QSOs (QSO$\alpha$), and other
spectral types will be supported.

The spectral templates are constructed using a combination of empirical
constraints (e.g., spectroscopy and photometry of galaxies and quasars over a
broad redshift range) and state-of-the-art theoretical models.  These templates
have a myriad of applications, but will be used primarily to help develop and
optimize the end-to-end DESI spectroscopic pipeline (``data challenges'')---from
raw data to redshifts and final spectral classifications---as well as to help
develop and validate various target-selection strategies.

All the code described below, as well as the text of this document and the code
used to generate the figures are maintained in the public DESI {\tt Github}
repository {\tt
  desihub/desisim}\footnote{\url{https://github.com/desihub/desisim}} under the
{\tt py/desisim} and {\tt doc/tex/simulate-templates} directories, respectively.

%Finally, everything all the code has been developed on {\tt Github} in {\tt
%  Python} using the same coding standards adopted by the {\tt DESI-Data} team,
%such that other members of the collaboration can contribute to its future
%development.   Lots of qaplots get written out and DESI color cuts are
%optionally written out!

\section{Getting Started Quickly}

This section presents several high-level examples so that you can get started
building templates quickly.  Before you try these examples please see
Appendix~\ref{sec:install} for instructions on how to install the software and
its dependencies.

The only mandatory input required by \simt{} is to specify the number of
templates via the {\tt nmodel} input and, optionally, the object type (although
the default is {\tt ELG}); the code is initialized with sensible defaults for
all the other optional inputs.  The following sections demonstrate how to build
templates for a few different object types.

\subsection{Example: ELGs}

To generate 20 {\tt ELG} spectra with the default parameters type:

\begin{verbatim}
simulate-templates.py --nmodel 100 --objtype ELG
\end{verbatim}

\subsection{Example: LRGs}

Give some examples.

\section{Implementation Details}

The following sections describe each specific DESI spectral class in more
detail, as well as various planned (and unplanned) avenues for improving the
models.

%\section{Building a Realistic Emission-Line Spectrum}

\subsection{Emission Line Galaxies (ELGs)}

\subsubsection{Background}

ELGs are characterized by nebular emission lines superposed on an underlying
stellar continuum.  The stellar continuum is typically blue, dominated by hot,
newly formed stars, although emission lines can be found in galaxies with more
evolved (i.e., older) stellar populations as well.  A nebular emission-line
spectrum consists of hydrogen and helium recombination lines (e.g., \ha, \hb,
\heli), as well as numerous forbidden metal-line transitions from singly and
doubly ionized oxygen (\oiilam{} and \oiiilam), nitrogen (\niilam), neon
(\neonlam), sulfur (\siilam), among many others.  The strengths of these lines
relative to \hb{} (or to one another) depends on the physical conditions (metal
abundance, ionizing radiation field, dust content, etc.) in the star-forming
\hii{} regions, and on the portion of the integrated light of the galaxy
captured by the spectroscopic observations.\footnote{This latter distinction is
  important because galaxies exhibit radial gradients in many of these
  properties, such as gas-phase abundance, leading to what is known as {\em
    aperture bias}.}  The absolute strength of an emission line is characterized
by its equivalent width, which is roughly given by the integrated line-flux
divided by the underlying stellar continuum flux.

A preliminary set of $7876$ ELG templates for DESI were developed in late-2014
using high-resolution spectroscopy and broadband photometry of galaxies from the
DEEP2 survey (see {\tt
  DESI-doc-870}).\footnote{\url{https://desi.lbl.gov/DocDB/cgi-bin/private/RetrieveFile?docid=870;filename=elg-templates.pdf;version=1}}
The current ELG templates build upon these templates but add significantly more
flexibility in how the nebular emission-line spectrum for each template is
constructed, as well as how that spectrum is linked to the underlying stellar
continuum spectrum.

\subsubsection{Methodology}

Generating templates for ELGs is challenging because the emission-line spectrum
of a galaxy cannot be constructed using theoretical models (i.e., from first
principles).\footnote{While photoionization models do exist, galaxies---and in
  particular the integrated spectra of galaxies---are too complex to model
  adequately using these models.}  Moreover, although they are often correlated,
the emission-line and stellar continuum spectra of a galaxy can exhibit large
variations due to stochastic star formation histories among star-forming
galaxies and differential dust attenuation effects.  Consequently, we have
developed an empirical approach for generating templates for ELGs.





Briefly, these templates were constructed by selecting a statistically complete
sample of DEEP2 galaxies at $z=0.75-1.45$ with $r_{\rm CFHTLS}<23.5$, a
well-measured \oiilam{} emission-line doublet, and at least six bands of optical
photometry \citep{newman13a, matthews13a}.  {\tt WISE} photometry at $3.4~\mu$m
(W1) and $4.6~\mu$m (W2) was included by running {\tt The
  Tractor}\footnote{\url{https://github.com/dstndstn/tractor.git}} in forced
photometry mode.  Next, the broadband optical and WISE photometry was fitted
using \isedfit, a Bayesian stellar population synthesis fitting code
\citep{moustakas13a}, to produce noiseless stellar continuum spectra from
$0.125-4.0~\mu$m (rest-frame) with $20$~\kms{} pixels for the whole sample.  The
observed-frame optical DEEP2 spectrum of each galaxy was also fitted in order to
measure the integrated flux, intrinsic velocity width (deconvolved for the DEEP2
instrumental resolution), and flux-ratio of the \oiilam{} emission-line doublet.
Additional nebular emission lines were generated in an ad hoc way (e.g., using
fixed emission-line ratios) and the continuum and emission-line model spectra
were combined into the \emph{preliminary} ELG templates described in {\tt
  DESI-doc-870}.

%intermediate-resolution ($\mathcal{R}\approx2200$) 

The method we use is
empirically calibrated using observed emission-line sequences and emission-line
equivalent width measurements of galaxies at $z=0-2$, ensuring that the ELG
templates exhibit the same range and variation in dust content, gas-phase
abundance, excitation, and specific star formation rate (star formation rate
divided by stellar mass) as real star-forming galaxies.

Second, we use a Monte Carlo approach so that an arbitrarily large number of
``random'' templates can be generated.  All the user has to specify is the
desired number of templates; all the other parameters, including the redshift
range, $r$-band magnitude range (which sets the overall normalization and
ultimately the integrated \oii{} emission-line flux), and emission-line priors
are drawn from appropriate prior distributions.  Alternatively, the user has
significant (optional) control over the input priors so that the templates can
be customized for a wide range of applications.  In addition, in order to
facilitate testing, the seed for the random number generator is an optional
input, making the ``random'' templates fully reproducible.

%For example, emission-line ratios can vary significantly (up to an
%order-of-magnitude or more) due to differences in dust content, gas-phase
%abundance, and excitation among star-forming galaxies
%\citep[e.g.,][]{kewley01a}.  And second, the strengths of the nebular emission
%lines relative to the underlying stellar continuum (i.e., the emission-line
%equivalent width) also exhibit large intrinsic variations due to differences in
%the specific star formation rate (star formation rate divided by the stellar
%mass; see \citealt{kennicutt94a}) among galaxies.  Moreover, we need to take
%into account the fact that DESI will be targeting ELGs in an epoch ($z\sim1$)
%when galaxies were dustier and forming stars at much higher rates (relative to
%their stellar mass) compared to local star-forming galaxies \citep{blanton09a,
%madau14a}.

%\begin{itemize*}
%\item{}
%\end{itemize*}

Each ELG template is generated according to the following procedure, which we
discuss in more detail below:

We have developed the following procedure for generating a broad range of
empirically constrained ELG templates:
\begin{enumerate*}
\item{Randomly (i.e., with equal probability) choose a DEEP2 stellar continuum
  spectrum with measured $4000$-\AA{} break strength $D_{n}(4000)$ (a proxy for
  specific star-formation rate).}
\item{Use an empirically calibrated relation between $D_{n}(4000)$ and
  rest-frame \ewoiilam{} to construct a suitably scaled \oii{} emission line.}
\item{}
\end{enumerate*}

\subsubsection{Planned or Possible Improvements}

Room for improvement: allow for different attenuation curves.  Should also
constrain the priors on the dust distribution better.

* Allow AGN-like emission-line ratios.
* Add (weak) emission lines to the LRG templates.
* Fix the random seed so the spectrum results are reproducible.
* Add units tests.
* Add nebular continuum to the EMSpectrum class.
* The empirical coefficients in EMSpectrum are hard-coded.
* Add more emission lines.
* allow redshift and r-band magnitudes to be drawn from a distribution

\begin{figure}
\centering
\includegraphics[width=0.8\textwidth]{figures/oiiihb.pdf}
\caption{Caption. \label{fig:oiiihb}}
\end{figure}

%DESI aims to measure redshifts for approximately $18$~million ELGs over the
%redshift range $0.6<z<1.6$, primarily by resolving the \oiilam{} doublet.

\begin{figure}
\centering
\includegraphics[width=0.8\textwidth]{figures/linesigma.pdf}
\caption{Distribution of emission-line velocity widths among DEEP2 galaxies at
  $z\sim1$.  The distribution is reasonably well-described by a log-normal
  distribution with mean and standard deviation of
  $\langle\log_{10}\sigma\rangle=1.877\pm0.175$~\kms{} or
  $\langle\sigma\rangle=77$~\kms.  \label{fig:linesigma}}
\end{figure}

\begin{figure}
\centering
\includegraphics[width=0.8\textwidth]{figures/d4000_ewoii.pdf}
\caption{Rest-frame \oii{} equivalent width vs.~the $4000$-\AA{}
  break, $D_{n}(4000)$.  The blue points and uncertainties indicate
  the running median and standard deviation in $0.05$-wide bins of
  $D_{n}(4000)$, while the red curve is a third-order polynomial fit
  to the points with the coefficients given in the
  legend.  The dispersion is relatively uniform with $D_{n}(4000)$
  with a mean value of $0.30$~dex.  {\bf Show the subset of objects
    that satisfy our grz color cuts!}  \label{fig:d4000}}
\end{figure}



\begin{enumerate}
\item{Extension~0: An [{\sc npixel}]$\times$[{\sc ntemplate}] flux
  array which contains {\sc ntemplate} spectra, each {\sc npixel}
  pixels long (see Table~\ref{table:rest} above).}
\item{Extension~1: An [{\sc ntemplate}] data structure with the
  following metadata:
\begin{itemize}
\item{.{\sc TEMPLATEID} - unique template ID number (zero-indexed)}
\end{itemize}
}
\end{enumerate}

The observed-frame template file, {\tt
  elg\_templates\_obs\_VERSION.fits.gz}, is a binary FITS table with
two extensions:

\begin{enumerate}
\item{Extension~0: An [{\sc npixel}]$\times$[{\sc ntemplate}] flux
  array which contains {\sc ntemplate} spectra, each {\sc npixel}
  pixels long (see Table~\ref{table:obs} above).}
\item{Extension~1: An [{\sc ntemplate}] data structure with the
  following metadata:
\begin{itemize}
\item{.{\sc TEMPLATEID} - unique template ID number (zero-indexed)}
\item{.{\sc OBJNO} - unique DEEP2 identifier}
\item{.{\sc RA} - right ascension (degrees, J2000)}
\item{.{\sc DEC} - declination (degrees, J2000)}
\item{.{\sc Z} - heliocentric redshift of the DEEP2 galaxy} 
\item{.{\sc SIGMA\_KMS} - intrinsic velocity line-width (km/s)} 
\item{.{\sc OII\_3726} - \oii~$\lambda3726$ emission-line flux
  (erg~s$^{-1}$~cm$^{^2}$)}  
\item{.{\sc OII\_3729} - \oii~$\lambda3729$ emission-line flux
  (erg~s$^{-1}$~cm$^{^2}$)} 
\item{.{\sc OII\_3727} - total \oiilam{} emission-line flux
  (erg~s$^{-1}$~cm$^{^2}$)} 
\item{.{\sc OII\_3727\_EW} - rest-frame \oii~$\lambda3726,29$
  equivalent width (\AA)}
\item{.{\sc RADIUS\_HALFLIGHT} - half-light radius (arcsec) ($-999$ if not measured)}
\item{.{\sc SERSICN} - Sersic n parameter (concentration) ($-999$ if not measured)}
\item{.{\sc AXIS\_RATIO} - minor-to-major axis ratio ($-999$ if not measured)}
\item{.{\sc DECAM\_G} - synthesized DECam g-band AB magnitude}
\item{.{\sc DECAM\_R} - synthesized DECam r-band AB magnitude}
\item{.{\sc DECAM\_Z} - synthesized DECam z-band AB magnitude}
\item{.{\sc LOGMSTAR} - log10(stellar mass) ($M_{\odot}$, $h=0.7$,
  \citealt{chabrier03a} IMF)} 
\item{.{\sc LOGSFR} - log10(star formation rate)
  ($M_{\odot}$~yr$^{-1}$, $h=0.7$, \citealt{chabrier03a} IMF)}
\item{.{\sc AV\_ISM} - $V$-band continuum attenuation (mag,
  \citealt{charlot00a} attenuation curve)} 
\end{itemize}
}
\end{enumerate}

\subsection{Luminous Red Galaxies (LRGs)}

LRGs


\subsection{Normal Stars (STAR)}

Talk about stuff.

\subsection{F-Type Standard Stars (FSTD)}

\subsection{White Dwarfs (WD)}

Stuff.

\subsection{Bright-Galaxy Survey Galaxies (BGS)}

BGS.  Not yet supported.

\subsection{Quasars (QSO)}

Lyman-alpha quasars.

\subsection{Future Improvements}

Future improvements to these templates may include (in no particular
order): 

\begin{enumerate}
\item{Extending the spectral coverage redward into the mid-infrared so
  that WISE photometry can be synthesized from the models.}
\item{Including additional nebular emission lines.}
\item{Constructing a smaller basis set of models (e.g., archetypes)
  which adequately span the physical parameter space of the full set
  of templates.}
\item{Incorporate more intelligent resampling of the data and models,
  in order to ensure that flux is being conserved correctly.}
\item{Expanding this TechNote to include more quantitative tests of
  the models (e.g., how well they span color-color and color-redshift
  space relative to actual observations).}
\end{enumerate}


\bibliographystyle{apj}
\bibliography{/Users/ioannis/bibdesk/ioannis}
%\documentclass[12pt]{article}
\usepackage{epsfig, mdwlist, amssymb}
\usepackage{fullpage}
%\usepackage{color}
\usepackage{url}
\usepackage{natbib}
%%\usepackage{emulateapj}
%\usepackage{footnote}
\usepackage{floatrow}
\floatsetup[table]{capposition=top}
\usepackage{deluxetable}

\newcommand{\ha}{H\ensuremath{\alpha}}
\newcommand{\hb}{H\ensuremath{\beta}}
\newcommand{\heli}{He~{\sc i}~\ensuremath{\lambda4471}}
\newcommand{\hei}{He~{\sc i}}
\newcommand{\hii}{H~{\sc ii}}

\newcommand{\oii}{[O~{\sc ii}]}
\newcommand{\oiii}{[O~{\sc iii}]}
\newcommand{\nii}{[N~{\sc ii}]}
\newcommand{\sii}{[N~{\sc ii}]}
\newcommand{\simt}{{\tt simulate-templates}}
\newcommand{\ewoii}{EW([O~{\sc ii}])}

\newcommand{\niilam}{[N~{\sc ii}]~\ensuremath{\lambda\lambda6548,84}}
\newcommand{\siilam}{[S~{\sc ii}]~\ensuremath{\lambda\lambda6716,32}}
\newcommand{\oiilam}{[O~{\sc ii}]~\ensuremath{\lambda\lambda3726,29}}
\newcommand{\oiiilam}{[O~{\sc iii}]~\ensuremath{\lambda\lambda4959,5007}}
\newcommand{\neonlam}{[Ne~{\sc iii}]~\ensuremath{\lambda3869}}

\newcommand{\ewoiilam}{EW([O~{\sc ii}]~\ensuremath{\lambda\lambda3726,29})}

\newcommand{\kms}{km~s$^{-1}$}
\newcommand{\isedfit}{{\tt iSEDfit}}

\newcommand{\apj}{ApJ}
\newcommand{\pasp}{PASP}
\newcommand{\araa}{ARA\&A}
\newcommand{\apjs}{ApJS}
\newcommand{\TODO}[1]{{\it \color{red} (#1)}}

\begin{document}
\title{Simulating State-of-the-Art Spectral Templates \\ for DESI} 

%{\bf TODO: Write a test script which ensures everything has been properly
%  installed/configured.}

\author{John Moustakas (Siena College)}
\maketitle

%\abstract{ This technote documents the construction of the preliminary
%  (currently v1.2) emission-line galaxy (ELG) templates for DESI.  }

\section{Introduction}

This document describes the methodology and code which was developed to generate
one-dimensional, noiseless, high-resolution spectral templates for DESI.  The
initial focus of this effort has been on emission-line galaxies (ELG), luminous
red galaxies (LRG), (normal) stars (STAR), including F-type standard stars
(FSTD), and white dwarfs (WD), but eventually bright-galaxy survey galaxies
(BGS), quasars (QSO), including Ly$\alpha$-QSOs (QSO$\alpha$), and other
spectral types will be supported.

The spectral templates are constructed using a combination of empirical
constraints (e.g., spectroscopy and photometry of galaxies and quasars over a
broad redshift range) and state-of-the-art theoretical models.  These templates
have a myriad of applications, but will be used primarily to help develop and
optimize the end-to-end DESI spectroscopic pipeline (``data challenges'')---from
raw data to redshifts and final spectral classifications---as well as to help
develop and validate various target-selection strategies.

All the code described below, as well as the text of this document and the code
used to generate the figures are maintained in the public DESI {\tt Github}
repository {\tt
  desihub/desisim}\footnote{\url{https://github.com/desihub/desisim}} under the
{\tt py/desisim} and {\tt doc/tex/simulate-templates} directories, respectively.

%Finally, everything all the code has been developed on {\tt Github} in {\tt
%  Python} using the same coding standards adopted by the {\tt DESI-Data} team,
%such that other members of the collaboration can contribute to its future
%development.   Lots of qaplots get written out and DESI color cuts are
%optionally written out!

\section{Getting Started Quickly}

This section presents several high-level examples so that you can get started
building templates quickly.  Before you try these examples please see
Appendix~\ref{sec:install} for instructions on how to install the software and
its dependencies.

The only mandatory input required by \simt{} is to specify the number of
templates via the {\tt nmodel} input and, optionally, the object type (although
the default is {\tt ELG}); the code is initialized with sensible defaults for
all the other optional inputs.  The following sections demonstrate how to build
templates for a few different object types.

\subsection{Example: ELGs}

To generate 20 {\tt ELG} spectra with the default parameters type:

\begin{verbatim}
simulate-templates.py --nmodel 100 --objtype ELG
\end{verbatim}

\subsection{Example: LRGs}

Give some examples.

\section{Implementation Details}

The following sections describe each specific DESI spectral class in more
detail, as well as various planned (and unplanned) avenues for improving the
models.

%\section{Building a Realistic Emission-Line Spectrum}

\subsection{Emission Line Galaxies (ELGs)}

\subsubsection{Background}

ELGs are characterized by nebular emission lines superposed on an underlying
stellar continuum.  The stellar continuum is typically blue, dominated by hot,
newly formed stars, although emission lines can be found in galaxies with more
evolved (i.e., older) stellar populations as well.  A nebular emission-line
spectrum consists of hydrogen and helium recombination lines (e.g., \ha, \hb,
\heli), as well as numerous forbidden metal-line transitions from singly and
doubly ionized oxygen (\oiilam{} and \oiiilam), nitrogen (\niilam), neon
(\neonlam), sulfur (\siilam), among many others.  The strengths of these lines
relative to \hb{} (or to one another) depends on the physical conditions (metal
abundance, ionizing radiation field, dust content, etc.) in the star-forming
\hii{} regions, and on the portion of the integrated light of the galaxy
captured by the spectroscopic observations.\footnote{This latter distinction is
  important because galaxies exhibit radial gradients in many of these
  properties, such as gas-phase abundance, leading to what is known as {\em
    aperture bias}.}  The absolute strength of an emission line is characterized
by its equivalent width, which is roughly given by the integrated line-flux
divided by the underlying stellar continuum flux.

A preliminary set of $7876$ ELG templates for DESI were developed in late-2014
using high-resolution spectroscopy and broadband photometry of galaxies from the
DEEP2 survey (see {\tt
  DESI-doc-870}).\footnote{\url{https://desi.lbl.gov/DocDB/cgi-bin/private/RetrieveFile?docid=870;filename=elg-templates.pdf;version=1}}
The current ELG templates build upon these templates but add significantly more
flexibility in how the nebular emission-line spectrum for each template is
constructed, as well as how that spectrum is linked to the underlying stellar
continuum spectrum.

\subsubsection{Methodology}

Generating templates for ELGs is challenging because the emission-line spectrum
of a galaxy cannot be constructed using theoretical models (i.e., from first
principles).\footnote{While photoionization models do exist, galaxies---and in
  particular the integrated spectra of galaxies---are too complex to model
  adequately using these models.}  Moreover, although they are often correlated,
the emission-line and stellar continuum spectra of a galaxy can exhibit large
variations due to stochastic star formation histories among star-forming
galaxies and differential dust attenuation effects.  Consequently, we have
developed an empirical approach for generating templates for ELGs.





Briefly, these templates were constructed by selecting a statistically complete
sample of DEEP2 galaxies at $z=0.75-1.45$ with $r_{\rm CFHTLS}<23.5$, a
well-measured \oiilam{} emission-line doublet, and at least six bands of optical
photometry \citep{newman13a, matthews13a}.  {\tt WISE} photometry at $3.4~\mu$m
(W1) and $4.6~\mu$m (W2) was included by running {\tt The
  Tractor}\footnote{\url{https://github.com/dstndstn/tractor.git}} in forced
photometry mode.  Next, the broadband optical and WISE photometry was fitted
using \isedfit, a Bayesian stellar population synthesis fitting code
\citep{moustakas13a}, to produce noiseless stellar continuum spectra from
$0.125-4.0~\mu$m (rest-frame) with $20$~\kms{} pixels for the whole sample.  The
observed-frame optical DEEP2 spectrum of each galaxy was also fitted in order to
measure the integrated flux, intrinsic velocity width (deconvolved for the DEEP2
instrumental resolution), and flux-ratio of the \oiilam{} emission-line doublet.
Additional nebular emission lines were generated in an ad hoc way (e.g., using
fixed emission-line ratios) and the continuum and emission-line model spectra
were combined into the \emph{preliminary} ELG templates described in {\tt
  DESI-doc-870}.

%intermediate-resolution ($\mathcal{R}\approx2200$) 

The method we use is
empirically calibrated using observed emission-line sequences and emission-line
equivalent width measurements of galaxies at $z=0-2$, ensuring that the ELG
templates exhibit the same range and variation in dust content, gas-phase
abundance, excitation, and specific star formation rate (star formation rate
divided by stellar mass) as real star-forming galaxies.

Second, we use a Monte Carlo approach so that an arbitrarily large number of
``random'' templates can be generated.  All the user has to specify is the
desired number of templates; all the other parameters, including the redshift
range, $r$-band magnitude range (which sets the overall normalization and
ultimately the integrated \oii{} emission-line flux), and emission-line priors
are drawn from appropriate prior distributions.  Alternatively, the user has
significant (optional) control over the input priors so that the templates can
be customized for a wide range of applications.  In addition, in order to
facilitate testing, the seed for the random number generator is an optional
input, making the ``random'' templates fully reproducible.

%For example, emission-line ratios can vary significantly (up to an
%order-of-magnitude or more) due to differences in dust content, gas-phase
%abundance, and excitation among star-forming galaxies
%\citep[e.g.,][]{kewley01a}.  And second, the strengths of the nebular emission
%lines relative to the underlying stellar continuum (i.e., the emission-line
%equivalent width) also exhibit large intrinsic variations due to differences in
%the specific star formation rate (star formation rate divided by the stellar
%mass; see \citealt{kennicutt94a}) among galaxies.  Moreover, we need to take
%into account the fact that DESI will be targeting ELGs in an epoch ($z\sim1$)
%when galaxies were dustier and forming stars at much higher rates (relative to
%their stellar mass) compared to local star-forming galaxies \citep{blanton09a,
%madau14a}.

%\begin{itemize*}
%\item{}
%\end{itemize*}

Each ELG template is generated according to the following procedure, which we
discuss in more detail below:

We have developed the following procedure for generating a broad range of
empirically constrained ELG templates:
\begin{enumerate*}
\item{Randomly (i.e., with equal probability) choose a DEEP2 stellar continuum
  spectrum with measured $4000$-\AA{} break strength $D_{n}(4000)$ (a proxy for
  specific star-formation rate).}
\item{Use an empirically calibrated relation between $D_{n}(4000)$ and
  rest-frame \ewoiilam{} to construct a suitably scaled \oii{} emission line.}
\item{}
\end{enumerate*}

\subsubsection{Planned or Possible Improvements}

Room for improvement: allow for different attenuation curves.  Should also
constrain the priors on the dust distribution better.

* Allow AGN-like emission-line ratios.
* Add (weak) emission lines to the LRG templates.
* Fix the random seed so the spectrum results are reproducible.
* Add units tests.
* Add nebular continuum to the EMSpectrum class.
* The empirical coefficients in EMSpectrum are hard-coded.
* Add more emission lines.
* allow redshift and r-band magnitudes to be drawn from a distribution

\begin{figure}
\centering
\includegraphics[width=0.8\textwidth]{figures/oiiihb.pdf}
\caption{Caption. \label{fig:oiiihb}}
\end{figure}

%DESI aims to measure redshifts for approximately $18$~million ELGs over the
%redshift range $0.6<z<1.6$, primarily by resolving the \oiilam{} doublet.

\begin{figure}
\centering
\includegraphics[width=0.8\textwidth]{figures/linesigma.pdf}
\caption{Distribution of emission-line velocity widths among DEEP2 galaxies at
  $z\sim1$.  The distribution is reasonably well-described by a log-normal
  distribution with mean and standard deviation of
  $\langle\log_{10}\sigma\rangle=1.877\pm0.175$~\kms{} or
  $\langle\sigma\rangle=77$~\kms.  \label{fig:linesigma}}
\end{figure}

\begin{figure}
\centering
\includegraphics[width=0.8\textwidth]{figures/d4000_ewoii.pdf}
\caption{Rest-frame \oii{} equivalent width vs.~the $4000$-\AA{}
  break, $D_{n}(4000)$.  The blue points and uncertainties indicate
  the running median and standard deviation in $0.05$-wide bins of
  $D_{n}(4000)$, while the red curve is a third-order polynomial fit
  to the points with the coefficients given in the
  legend.  The dispersion is relatively uniform with $D_{n}(4000)$
  with a mean value of $0.30$~dex.  {\bf Show the subset of objects
    that satisfy our grz color cuts!}  \label{fig:d4000}}
\end{figure}



\begin{enumerate}
\item{Extension~0: An [{\sc npixel}]$\times$[{\sc ntemplate}] flux
  array which contains {\sc ntemplate} spectra, each {\sc npixel}
  pixels long (see Table~\ref{table:rest} above).}
\item{Extension~1: An [{\sc ntemplate}] data structure with the
  following metadata:
\begin{itemize}
\item{.{\sc TEMPLATEID} - unique template ID number (zero-indexed)}
\end{itemize}
}
\end{enumerate}

The observed-frame template file, {\tt
  elg\_templates\_obs\_VERSION.fits.gz}, is a binary FITS table with
two extensions:

\begin{enumerate}
\item{Extension~0: An [{\sc npixel}]$\times$[{\sc ntemplate}] flux
  array which contains {\sc ntemplate} spectra, each {\sc npixel}
  pixels long (see Table~\ref{table:obs} above).}
\item{Extension~1: An [{\sc ntemplate}] data structure with the
  following metadata:
\begin{itemize}
\item{.{\sc TEMPLATEID} - unique template ID number (zero-indexed)}
\item{.{\sc OBJNO} - unique DEEP2 identifier}
\item{.{\sc RA} - right ascension (degrees, J2000)}
\item{.{\sc DEC} - declination (degrees, J2000)}
\item{.{\sc Z} - heliocentric redshift of the DEEP2 galaxy} 
\item{.{\sc SIGMA\_KMS} - intrinsic velocity line-width (km/s)} 
\item{.{\sc OII\_3726} - \oii~$\lambda3726$ emission-line flux
  (erg~s$^{-1}$~cm$^{^2}$)}  
\item{.{\sc OII\_3729} - \oii~$\lambda3729$ emission-line flux
  (erg~s$^{-1}$~cm$^{^2}$)} 
\item{.{\sc OII\_3727} - total \oiilam{} emission-line flux
  (erg~s$^{-1}$~cm$^{^2}$)} 
\item{.{\sc OII\_3727\_EW} - rest-frame \oii~$\lambda3726,29$
  equivalent width (\AA)}
\item{.{\sc RADIUS\_HALFLIGHT} - half-light radius (arcsec) ($-999$ if not measured)}
\item{.{\sc SERSICN} - Sersic n parameter (concentration) ($-999$ if not measured)}
\item{.{\sc AXIS\_RATIO} - minor-to-major axis ratio ($-999$ if not measured)}
\item{.{\sc DECAM\_G} - synthesized DECam g-band AB magnitude}
\item{.{\sc DECAM\_R} - synthesized DECam r-band AB magnitude}
\item{.{\sc DECAM\_Z} - synthesized DECam z-band AB magnitude}
\item{.{\sc LOGMSTAR} - log10(stellar mass) ($M_{\odot}$, $h=0.7$,
  \citealt{chabrier03a} IMF)} 
\item{.{\sc LOGSFR} - log10(star formation rate)
  ($M_{\odot}$~yr$^{-1}$, $h=0.7$, \citealt{chabrier03a} IMF)}
\item{.{\sc AV\_ISM} - $V$-band continuum attenuation (mag,
  \citealt{charlot00a} attenuation curve)} 
\end{itemize}
}
\end{enumerate}

\subsection{Luminous Red Galaxies (LRGs)}

LRGs


\subsection{Normal Stars (STAR)}

Talk about stuff.

\subsection{F-Type Standard Stars (FSTD)}

\subsection{White Dwarfs (WD)}

Stuff.

\subsection{Bright-Galaxy Survey Galaxies (BGS)}

BGS.  Not yet supported.

\subsection{Quasars (QSO)}

Lyman-alpha quasars.

\subsection{Future Improvements}

Future improvements to these templates may include (in no particular
order): 

\begin{enumerate}
\item{Extending the spectral coverage redward into the mid-infrared so
  that WISE photometry can be synthesized from the models.}
\item{Including additional nebular emission lines.}
\item{Constructing a smaller basis set of models (e.g., archetypes)
  which adequately span the physical parameter space of the full set
  of templates.}
\item{Incorporate more intelligent resampling of the data and models,
  in order to ensure that flux is being conserved correctly.}
\item{Expanding this TechNote to include more quantitative tests of
  the models (e.g., how well they span color-color and color-redshift
  space relative to actual observations).}
\end{enumerate}


\bibliographystyle{apj}
\bibliography{/Users/ioannis/bibdesk/ioannis}
%\documentclass[12pt]{article}
\usepackage{epsfig, mdwlist, amssymb}
\usepackage{fullpage}
%\usepackage{color}
\usepackage{url}
\usepackage{natbib}
%%\usepackage{emulateapj}
%\usepackage{footnote}
\usepackage{floatrow}
\floatsetup[table]{capposition=top}
\usepackage{deluxetable}

\newcommand{\ha}{H\ensuremath{\alpha}}
\newcommand{\hb}{H\ensuremath{\beta}}
\newcommand{\heli}{He~{\sc i}~\ensuremath{\lambda4471}}
\newcommand{\hei}{He~{\sc i}}
\newcommand{\hii}{H~{\sc ii}}

\newcommand{\oii}{[O~{\sc ii}]}
\newcommand{\oiii}{[O~{\sc iii}]}
\newcommand{\nii}{[N~{\sc ii}]}
\newcommand{\sii}{[N~{\sc ii}]}
\newcommand{\simt}{{\tt simulate-templates}}
\newcommand{\ewoii}{EW([O~{\sc ii}])}

\newcommand{\niilam}{[N~{\sc ii}]~\ensuremath{\lambda\lambda6548,84}}
\newcommand{\siilam}{[S~{\sc ii}]~\ensuremath{\lambda\lambda6716,32}}
\newcommand{\oiilam}{[O~{\sc ii}]~\ensuremath{\lambda\lambda3726,29}}
\newcommand{\oiiilam}{[O~{\sc iii}]~\ensuremath{\lambda\lambda4959,5007}}
\newcommand{\neonlam}{[Ne~{\sc iii}]~\ensuremath{\lambda3869}}

\newcommand{\ewoiilam}{EW([O~{\sc ii}]~\ensuremath{\lambda\lambda3726,29})}

\newcommand{\kms}{km~s$^{-1}$}
\newcommand{\isedfit}{{\tt iSEDfit}}

\newcommand{\apj}{ApJ}
\newcommand{\pasp}{PASP}
\newcommand{\araa}{ARA\&A}
\newcommand{\apjs}{ApJS}
\newcommand{\TODO}[1]{{\it \color{red} (#1)}}

\begin{document}
\title{Simulating State-of-the-Art Spectral Templates \\ for DESI} 

%{\bf TODO: Write a test script which ensures everything has been properly
%  installed/configured.}

\author{John Moustakas (Siena College)}
\maketitle

%\abstract{ This technote documents the construction of the preliminary
%  (currently v1.2) emission-line galaxy (ELG) templates for DESI.  }

\section{Introduction}

This document describes the methodology and code which was developed to generate
one-dimensional, noiseless, high-resolution spectral templates for DESI.  The
initial focus of this effort has been on emission-line galaxies (ELG), luminous
red galaxies (LRG), (normal) stars (STAR), including F-type standard stars
(FSTD), and white dwarfs (WD), but eventually bright-galaxy survey galaxies
(BGS), quasars (QSO), including Ly$\alpha$-QSOs (QSO$\alpha$), and other
spectral types will be supported.

The spectral templates are constructed using a combination of empirical
constraints (e.g., spectroscopy and photometry of galaxies and quasars over a
broad redshift range) and state-of-the-art theoretical models.  These templates
have a myriad of applications, but will be used primarily to help develop and
optimize the end-to-end DESI spectroscopic pipeline (``data challenges'')---from
raw data to redshifts and final spectral classifications---as well as to help
develop and validate various target-selection strategies.

All the code described below, as well as the text of this document and the code
used to generate the figures are maintained in the public DESI {\tt Github}
repository {\tt
  desihub/desisim}\footnote{\url{https://github.com/desihub/desisim}} under the
{\tt py/desisim} and {\tt doc/tex/simulate-templates} directories, respectively.

%Finally, everything all the code has been developed on {\tt Github} in {\tt
%  Python} using the same coding standards adopted by the {\tt DESI-Data} team,
%such that other members of the collaboration can contribute to its future
%development.   Lots of qaplots get written out and DESI color cuts are
%optionally written out!

\section{Getting Started Quickly}

This section presents several high-level examples so that you can get started
building templates quickly.  Before you try these examples please see
Appendix~\ref{sec:install} for instructions on how to install the software and
its dependencies.

The only mandatory input required by \simt{} is to specify the number of
templates via the {\tt nmodel} input and, optionally, the object type (although
the default is {\tt ELG}); the code is initialized with sensible defaults for
all the other optional inputs.  The following sections demonstrate how to build
templates for a few different object types.

\subsection{Example: ELGs}

To generate 20 {\tt ELG} spectra with the default parameters type:

\begin{verbatim}
simulate-templates.py --nmodel 100 --objtype ELG
\end{verbatim}

\subsection{Example: LRGs}

Give some examples.

\section{Implementation Details}

The following sections describe each specific DESI spectral class in more
detail, as well as various planned (and unplanned) avenues for improving the
models.

%\section{Building a Realistic Emission-Line Spectrum}

\subsection{Emission Line Galaxies (ELGs)}

\subsubsection{Background}

ELGs are characterized by nebular emission lines superposed on an underlying
stellar continuum.  The stellar continuum is typically blue, dominated by hot,
newly formed stars, although emission lines can be found in galaxies with more
evolved (i.e., older) stellar populations as well.  A nebular emission-line
spectrum consists of hydrogen and helium recombination lines (e.g., \ha, \hb,
\heli), as well as numerous forbidden metal-line transitions from singly and
doubly ionized oxygen (\oiilam{} and \oiiilam), nitrogen (\niilam), neon
(\neonlam), sulfur (\siilam), among many others.  The strengths of these lines
relative to \hb{} (or to one another) depends on the physical conditions (metal
abundance, ionizing radiation field, dust content, etc.) in the star-forming
\hii{} regions, and on the portion of the integrated light of the galaxy
captured by the spectroscopic observations.\footnote{This latter distinction is
  important because galaxies exhibit radial gradients in many of these
  properties, such as gas-phase abundance, leading to what is known as {\em
    aperture bias}.}  The absolute strength of an emission line is characterized
by its equivalent width, which is roughly given by the integrated line-flux
divided by the underlying stellar continuum flux.

A preliminary set of $7876$ ELG templates for DESI were developed in late-2014
using high-resolution spectroscopy and broadband photometry of galaxies from the
DEEP2 survey (see {\tt
  DESI-doc-870}).\footnote{\url{https://desi.lbl.gov/DocDB/cgi-bin/private/RetrieveFile?docid=870;filename=elg-templates.pdf;version=1}}
The current ELG templates build upon these templates but add significantly more
flexibility in how the nebular emission-line spectrum for each template is
constructed, as well as how that spectrum is linked to the underlying stellar
continuum spectrum.

\subsubsection{Methodology}

Generating templates for ELGs is challenging because the emission-line spectrum
of a galaxy cannot be constructed using theoretical models (i.e., from first
principles).\footnote{While photoionization models do exist, galaxies---and in
  particular the integrated spectra of galaxies---are too complex to model
  adequately using these models.}  Moreover, although they are often correlated,
the emission-line and stellar continuum spectra of a galaxy can exhibit large
variations due to stochastic star formation histories among star-forming
galaxies and differential dust attenuation effects.  Consequently, we have
developed an empirical approach for generating templates for ELGs.





Briefly, these templates were constructed by selecting a statistically complete
sample of DEEP2 galaxies at $z=0.75-1.45$ with $r_{\rm CFHTLS}<23.5$, a
well-measured \oiilam{} emission-line doublet, and at least six bands of optical
photometry \citep{newman13a, matthews13a}.  {\tt WISE} photometry at $3.4~\mu$m
(W1) and $4.6~\mu$m (W2) was included by running {\tt The
  Tractor}\footnote{\url{https://github.com/dstndstn/tractor.git}} in forced
photometry mode.  Next, the broadband optical and WISE photometry was fitted
using \isedfit, a Bayesian stellar population synthesis fitting code
\citep{moustakas13a}, to produce noiseless stellar continuum spectra from
$0.125-4.0~\mu$m (rest-frame) with $20$~\kms{} pixels for the whole sample.  The
observed-frame optical DEEP2 spectrum of each galaxy was also fitted in order to
measure the integrated flux, intrinsic velocity width (deconvolved for the DEEP2
instrumental resolution), and flux-ratio of the \oiilam{} emission-line doublet.
Additional nebular emission lines were generated in an ad hoc way (e.g., using
fixed emission-line ratios) and the continuum and emission-line model spectra
were combined into the \emph{preliminary} ELG templates described in {\tt
  DESI-doc-870}.

%intermediate-resolution ($\mathcal{R}\approx2200$) 

The method we use is
empirically calibrated using observed emission-line sequences and emission-line
equivalent width measurements of galaxies at $z=0-2$, ensuring that the ELG
templates exhibit the same range and variation in dust content, gas-phase
abundance, excitation, and specific star formation rate (star formation rate
divided by stellar mass) as real star-forming galaxies.

Second, we use a Monte Carlo approach so that an arbitrarily large number of
``random'' templates can be generated.  All the user has to specify is the
desired number of templates; all the other parameters, including the redshift
range, $r$-band magnitude range (which sets the overall normalization and
ultimately the integrated \oii{} emission-line flux), and emission-line priors
are drawn from appropriate prior distributions.  Alternatively, the user has
significant (optional) control over the input priors so that the templates can
be customized for a wide range of applications.  In addition, in order to
facilitate testing, the seed for the random number generator is an optional
input, making the ``random'' templates fully reproducible.

%For example, emission-line ratios can vary significantly (up to an
%order-of-magnitude or more) due to differences in dust content, gas-phase
%abundance, and excitation among star-forming galaxies
%\citep[e.g.,][]{kewley01a}.  And second, the strengths of the nebular emission
%lines relative to the underlying stellar continuum (i.e., the emission-line
%equivalent width) also exhibit large intrinsic variations due to differences in
%the specific star formation rate (star formation rate divided by the stellar
%mass; see \citealt{kennicutt94a}) among galaxies.  Moreover, we need to take
%into account the fact that DESI will be targeting ELGs in an epoch ($z\sim1$)
%when galaxies were dustier and forming stars at much higher rates (relative to
%their stellar mass) compared to local star-forming galaxies \citep{blanton09a,
%madau14a}.

%\begin{itemize*}
%\item{}
%\end{itemize*}

Each ELG template is generated according to the following procedure, which we
discuss in more detail below:

We have developed the following procedure for generating a broad range of
empirically constrained ELG templates:
\begin{enumerate*}
\item{Randomly (i.e., with equal probability) choose a DEEP2 stellar continuum
  spectrum with measured $4000$-\AA{} break strength $D_{n}(4000)$ (a proxy for
  specific star-formation rate).}
\item{Use an empirically calibrated relation between $D_{n}(4000)$ and
  rest-frame \ewoiilam{} to construct a suitably scaled \oii{} emission line.}
\item{}
\end{enumerate*}

\subsubsection{Planned or Possible Improvements}

Room for improvement: allow for different attenuation curves.  Should also
constrain the priors on the dust distribution better.

* Allow AGN-like emission-line ratios.
* Add (weak) emission lines to the LRG templates.
* Fix the random seed so the spectrum results are reproducible.
* Add units tests.
* Add nebular continuum to the EMSpectrum class.
* The empirical coefficients in EMSpectrum are hard-coded.
* Add more emission lines.
* allow redshift and r-band magnitudes to be drawn from a distribution

\begin{figure}
\centering
\includegraphics[width=0.8\textwidth]{figures/oiiihb.pdf}
\caption{Caption. \label{fig:oiiihb}}
\end{figure}

%DESI aims to measure redshifts for approximately $18$~million ELGs over the
%redshift range $0.6<z<1.6$, primarily by resolving the \oiilam{} doublet.

\begin{figure}
\centering
\includegraphics[width=0.8\textwidth]{figures/linesigma.pdf}
\caption{Distribution of emission-line velocity widths among DEEP2 galaxies at
  $z\sim1$.  The distribution is reasonably well-described by a log-normal
  distribution with mean and standard deviation of
  $\langle\log_{10}\sigma\rangle=1.877\pm0.175$~\kms{} or
  $\langle\sigma\rangle=77$~\kms.  \label{fig:linesigma}}
\end{figure}

\begin{figure}
\centering
\includegraphics[width=0.8\textwidth]{figures/d4000_ewoii.pdf}
\caption{Rest-frame \oii{} equivalent width vs.~the $4000$-\AA{}
  break, $D_{n}(4000)$.  The blue points and uncertainties indicate
  the running median and standard deviation in $0.05$-wide bins of
  $D_{n}(4000)$, while the red curve is a third-order polynomial fit
  to the points with the coefficients given in the
  legend.  The dispersion is relatively uniform with $D_{n}(4000)$
  with a mean value of $0.30$~dex.  {\bf Show the subset of objects
    that satisfy our grz color cuts!}  \label{fig:d4000}}
\end{figure}



\begin{enumerate}
\item{Extension~0: An [{\sc npixel}]$\times$[{\sc ntemplate}] flux
  array which contains {\sc ntemplate} spectra, each {\sc npixel}
  pixels long (see Table~\ref{table:rest} above).}
\item{Extension~1: An [{\sc ntemplate}] data structure with the
  following metadata:
\begin{itemize}
\item{.{\sc TEMPLATEID} - unique template ID number (zero-indexed)}
\end{itemize}
}
\end{enumerate}

The observed-frame template file, {\tt
  elg\_templates\_obs\_VERSION.fits.gz}, is a binary FITS table with
two extensions:

\begin{enumerate}
\item{Extension~0: An [{\sc npixel}]$\times$[{\sc ntemplate}] flux
  array which contains {\sc ntemplate} spectra, each {\sc npixel}
  pixels long (see Table~\ref{table:obs} above).}
\item{Extension~1: An [{\sc ntemplate}] data structure with the
  following metadata:
\begin{itemize}
\item{.{\sc TEMPLATEID} - unique template ID number (zero-indexed)}
\item{.{\sc OBJNO} - unique DEEP2 identifier}
\item{.{\sc RA} - right ascension (degrees, J2000)}
\item{.{\sc DEC} - declination (degrees, J2000)}
\item{.{\sc Z} - heliocentric redshift of the DEEP2 galaxy} 
\item{.{\sc SIGMA\_KMS} - intrinsic velocity line-width (km/s)} 
\item{.{\sc OII\_3726} - \oii~$\lambda3726$ emission-line flux
  (erg~s$^{-1}$~cm$^{^2}$)}  
\item{.{\sc OII\_3729} - \oii~$\lambda3729$ emission-line flux
  (erg~s$^{-1}$~cm$^{^2}$)} 
\item{.{\sc OII\_3727} - total \oiilam{} emission-line flux
  (erg~s$^{-1}$~cm$^{^2}$)} 
\item{.{\sc OII\_3727\_EW} - rest-frame \oii~$\lambda3726,29$
  equivalent width (\AA)}
\item{.{\sc RADIUS\_HALFLIGHT} - half-light radius (arcsec) ($-999$ if not measured)}
\item{.{\sc SERSICN} - Sersic n parameter (concentration) ($-999$ if not measured)}
\item{.{\sc AXIS\_RATIO} - minor-to-major axis ratio ($-999$ if not measured)}
\item{.{\sc DECAM\_G} - synthesized DECam g-band AB magnitude}
\item{.{\sc DECAM\_R} - synthesized DECam r-band AB magnitude}
\item{.{\sc DECAM\_Z} - synthesized DECam z-band AB magnitude}
\item{.{\sc LOGMSTAR} - log10(stellar mass) ($M_{\odot}$, $h=0.7$,
  \citealt{chabrier03a} IMF)} 
\item{.{\sc LOGSFR} - log10(star formation rate)
  ($M_{\odot}$~yr$^{-1}$, $h=0.7$, \citealt{chabrier03a} IMF)}
\item{.{\sc AV\_ISM} - $V$-band continuum attenuation (mag,
  \citealt{charlot00a} attenuation curve)} 
\end{itemize}
}
\end{enumerate}

\subsection{Luminous Red Galaxies (LRGs)}

LRGs


\subsection{Normal Stars (STAR)}

Talk about stuff.

\subsection{F-Type Standard Stars (FSTD)}

\subsection{White Dwarfs (WD)}

Stuff.

\subsection{Bright-Galaxy Survey Galaxies (BGS)}

BGS.  Not yet supported.

\subsection{Quasars (QSO)}

Lyman-alpha quasars.

\subsection{Future Improvements}

Future improvements to these templates may include (in no particular
order): 

\begin{enumerate}
\item{Extending the spectral coverage redward into the mid-infrared so
  that WISE photometry can be synthesized from the models.}
\item{Including additional nebular emission lines.}
\item{Constructing a smaller basis set of models (e.g., archetypes)
  which adequately span the physical parameter space of the full set
  of templates.}
\item{Incorporate more intelligent resampling of the data and models,
  in order to ensure that flux is being conserved correctly.}
\item{Expanding this TechNote to include more quantitative tests of
  the models (e.g., how well they span color-color and color-redshift
  space relative to actual observations).}
\end{enumerate}


\bibliographystyle{apj}
\bibliography{/Users/ioannis/bibdesk/ioannis}
%\documentclass[12pt]{article}
\usepackage{epsfig, mdwlist, amssymb}
\usepackage{fullpage}
%\usepackage{color}
\usepackage{url}
\usepackage{natbib}
%%\usepackage{emulateapj}
%\usepackage{footnote}
\usepackage{floatrow}
\floatsetup[table]{capposition=top}
\usepackage{deluxetable}

\newcommand{\ha}{H\ensuremath{\alpha}}
\newcommand{\hb}{H\ensuremath{\beta}}
\newcommand{\heli}{He~{\sc i}~\ensuremath{\lambda4471}}
\newcommand{\hei}{He~{\sc i}}
\newcommand{\hii}{H~{\sc ii}}

\newcommand{\oii}{[O~{\sc ii}]}
\newcommand{\oiii}{[O~{\sc iii}]}
\newcommand{\nii}{[N~{\sc ii}]}
\newcommand{\sii}{[N~{\sc ii}]}
\newcommand{\simt}{{\tt simulate-templates}}
\newcommand{\ewoii}{EW([O~{\sc ii}])}

\newcommand{\niilam}{[N~{\sc ii}]~\ensuremath{\lambda\lambda6548,84}}
\newcommand{\siilam}{[S~{\sc ii}]~\ensuremath{\lambda\lambda6716,32}}
\newcommand{\oiilam}{[O~{\sc ii}]~\ensuremath{\lambda\lambda3726,29}}
\newcommand{\oiiilam}{[O~{\sc iii}]~\ensuremath{\lambda\lambda4959,5007}}
\newcommand{\neonlam}{[Ne~{\sc iii}]~\ensuremath{\lambda3869}}

\newcommand{\ewoiilam}{EW([O~{\sc ii}]~\ensuremath{\lambda\lambda3726,29})}

\newcommand{\kms}{km~s$^{-1}$}
\newcommand{\isedfit}{{\tt iSEDfit}}

\newcommand{\apj}{ApJ}
\newcommand{\pasp}{PASP}
\newcommand{\araa}{ARA\&A}
\newcommand{\apjs}{ApJS}
\newcommand{\TODO}[1]{{\it \color{red} (#1)}}

\begin{document}
\title{Simulating State-of-the-Art Spectral Templates \\ for DESI} 

%{\bf TODO: Write a test script which ensures everything has been properly
%  installed/configured.}

\author{John Moustakas (Siena College)}
\maketitle

%\abstract{ This technote documents the construction of the preliminary
%  (currently v1.2) emission-line galaxy (ELG) templates for DESI.  }

\section{Introduction}

This document describes the methodology and code which was developed to generate
one-dimensional, noiseless, high-resolution spectral templates for DESI.  The
initial focus of this effort has been on emission-line galaxies (ELG), luminous
red galaxies (LRG), (normal) stars (STAR), including F-type standard stars
(FSTD), and white dwarfs (WD), but eventually bright-galaxy survey galaxies
(BGS), quasars (QSO), including Ly$\alpha$-QSOs (QSO$\alpha$), and other
spectral types will be supported.

The spectral templates are constructed using a combination of empirical
constraints (e.g., spectroscopy and photometry of galaxies and quasars over a
broad redshift range) and state-of-the-art theoretical models.  These templates
have a myriad of applications, but will be used primarily to help develop and
optimize the end-to-end DESI spectroscopic pipeline (``data challenges'')---from
raw data to redshifts and final spectral classifications---as well as to help
develop and validate various target-selection strategies.

All the code described below, as well as the text of this document and the code
used to generate the figures are maintained in the public DESI {\tt Github}
repository {\tt
  desihub/desisim}\footnote{\url{https://github.com/desihub/desisim}} under the
{\tt py/desisim} and {\tt doc/tex/simulate-templates} directories, respectively.

%Finally, everything all the code has been developed on {\tt Github} in {\tt
%  Python} using the same coding standards adopted by the {\tt DESI-Data} team,
%such that other members of the collaboration can contribute to its future
%development.   Lots of qaplots get written out and DESI color cuts are
%optionally written out!

\section{Getting Started Quickly}

This section presents several high-level examples so that you can get started
building templates quickly.  Before you try these examples please see
Appendix~\ref{sec:install} for instructions on how to install the software and
its dependencies.

The only mandatory input required by \simt{} is to specify the number of
templates via the {\tt nmodel} input and, optionally, the object type (although
the default is {\tt ELG}); the code is initialized with sensible defaults for
all the other optional inputs.  The following sections demonstrate how to build
templates for a few different object types.

\subsection{Example: ELGs}

To generate 20 {\tt ELG} spectra with the default parameters type:

\begin{verbatim}
simulate-templates.py --nmodel 100 --objtype ELG
\end{verbatim}

\subsection{Example: LRGs}

Give some examples.

\section{Implementation Details}

The following sections describe each specific DESI spectral class in more
detail, as well as various planned (and unplanned) avenues for improving the
models.

%\section{Building a Realistic Emission-Line Spectrum}

\subsection{Emission Line Galaxies (ELGs)}

\subsubsection{Background}

ELGs are characterized by nebular emission lines superposed on an underlying
stellar continuum.  The stellar continuum is typically blue, dominated by hot,
newly formed stars, although emission lines can be found in galaxies with more
evolved (i.e., older) stellar populations as well.  A nebular emission-line
spectrum consists of hydrogen and helium recombination lines (e.g., \ha, \hb,
\heli), as well as numerous forbidden metal-line transitions from singly and
doubly ionized oxygen (\oiilam{} and \oiiilam), nitrogen (\niilam), neon
(\neonlam), sulfur (\siilam), among many others.  The strengths of these lines
relative to \hb{} (or to one another) depends on the physical conditions (metal
abundance, ionizing radiation field, dust content, etc.) in the star-forming
\hii{} regions, and on the portion of the integrated light of the galaxy
captured by the spectroscopic observations.\footnote{This latter distinction is
  important because galaxies exhibit radial gradients in many of these
  properties, such as gas-phase abundance, leading to what is known as {\em
    aperture bias}.}  The absolute strength of an emission line is characterized
by its equivalent width, which is roughly given by the integrated line-flux
divided by the underlying stellar continuum flux.

A preliminary set of $7876$ ELG templates for DESI were developed in late-2014
using high-resolution spectroscopy and broadband photometry of galaxies from the
DEEP2 survey (see {\tt
  DESI-doc-870}).\footnote{\url{https://desi.lbl.gov/DocDB/cgi-bin/private/RetrieveFile?docid=870;filename=elg-templates.pdf;version=1}}
The current ELG templates build upon these templates but add significantly more
flexibility in how the nebular emission-line spectrum for each template is
constructed, as well as how that spectrum is linked to the underlying stellar
continuum spectrum.

\subsubsection{Methodology}

Generating templates for ELGs is challenging because the emission-line spectrum
of a galaxy cannot be constructed using theoretical models (i.e., from first
principles).\footnote{While photoionization models do exist, galaxies---and in
  particular the integrated spectra of galaxies---are too complex to model
  adequately using these models.}  Moreover, although they are often correlated,
the emission-line and stellar continuum spectra of a galaxy can exhibit large
variations due to stochastic star formation histories among star-forming
galaxies and differential dust attenuation effects.  Consequently, we have
developed an empirical approach for generating templates for ELGs.





Briefly, these templates were constructed by selecting a statistically complete
sample of DEEP2 galaxies at $z=0.75-1.45$ with $r_{\rm CFHTLS}<23.5$, a
well-measured \oiilam{} emission-line doublet, and at least six bands of optical
photometry \citep{newman13a, matthews13a}.  {\tt WISE} photometry at $3.4~\mu$m
(W1) and $4.6~\mu$m (W2) was included by running {\tt The
  Tractor}\footnote{\url{https://github.com/dstndstn/tractor.git}} in forced
photometry mode.  Next, the broadband optical and WISE photometry was fitted
using \isedfit, a Bayesian stellar population synthesis fitting code
\citep{moustakas13a}, to produce noiseless stellar continuum spectra from
$0.125-4.0~\mu$m (rest-frame) with $20$~\kms{} pixels for the whole sample.  The
observed-frame optical DEEP2 spectrum of each galaxy was also fitted in order to
measure the integrated flux, intrinsic velocity width (deconvolved for the DEEP2
instrumental resolution), and flux-ratio of the \oiilam{} emission-line doublet.
Additional nebular emission lines were generated in an ad hoc way (e.g., using
fixed emission-line ratios) and the continuum and emission-line model spectra
were combined into the \emph{preliminary} ELG templates described in {\tt
  DESI-doc-870}.

%intermediate-resolution ($\mathcal{R}\approx2200$) 

The method we use is
empirically calibrated using observed emission-line sequences and emission-line
equivalent width measurements of galaxies at $z=0-2$, ensuring that the ELG
templates exhibit the same range and variation in dust content, gas-phase
abundance, excitation, and specific star formation rate (star formation rate
divided by stellar mass) as real star-forming galaxies.

Second, we use a Monte Carlo approach so that an arbitrarily large number of
``random'' templates can be generated.  All the user has to specify is the
desired number of templates; all the other parameters, including the redshift
range, $r$-band magnitude range (which sets the overall normalization and
ultimately the integrated \oii{} emission-line flux), and emission-line priors
are drawn from appropriate prior distributions.  Alternatively, the user has
significant (optional) control over the input priors so that the templates can
be customized for a wide range of applications.  In addition, in order to
facilitate testing, the seed for the random number generator is an optional
input, making the ``random'' templates fully reproducible.

%For example, emission-line ratios can vary significantly (up to an
%order-of-magnitude or more) due to differences in dust content, gas-phase
%abundance, and excitation among star-forming galaxies
%\citep[e.g.,][]{kewley01a}.  And second, the strengths of the nebular emission
%lines relative to the underlying stellar continuum (i.e., the emission-line
%equivalent width) also exhibit large intrinsic variations due to differences in
%the specific star formation rate (star formation rate divided by the stellar
%mass; see \citealt{kennicutt94a}) among galaxies.  Moreover, we need to take
%into account the fact that DESI will be targeting ELGs in an epoch ($z\sim1$)
%when galaxies were dustier and forming stars at much higher rates (relative to
%their stellar mass) compared to local star-forming galaxies \citep{blanton09a,
%madau14a}.

%\begin{itemize*}
%\item{}
%\end{itemize*}

Each ELG template is generated according to the following procedure, which we
discuss in more detail below:

We have developed the following procedure for generating a broad range of
empirically constrained ELG templates:
\begin{enumerate*}
\item{Randomly (i.e., with equal probability) choose a DEEP2 stellar continuum
  spectrum with measured $4000$-\AA{} break strength $D_{n}(4000)$ (a proxy for
  specific star-formation rate).}
\item{Use an empirically calibrated relation between $D_{n}(4000)$ and
  rest-frame \ewoiilam{} to construct a suitably scaled \oii{} emission line.}
\item{}
\end{enumerate*}

\subsubsection{Planned or Possible Improvements}

Room for improvement: allow for different attenuation curves.  Should also
constrain the priors on the dust distribution better.

* Allow AGN-like emission-line ratios.
* Add (weak) emission lines to the LRG templates.
* Fix the random seed so the spectrum results are reproducible.
* Add units tests.
* Add nebular continuum to the EMSpectrum class.
* The empirical coefficients in EMSpectrum are hard-coded.
* Add more emission lines.
* allow redshift and r-band magnitudes to be drawn from a distribution

\begin{figure}
\centering
\includegraphics[width=0.8\textwidth]{figures/oiiihb.pdf}
\caption{Caption. \label{fig:oiiihb}}
\end{figure}

%DESI aims to measure redshifts for approximately $18$~million ELGs over the
%redshift range $0.6<z<1.6$, primarily by resolving the \oiilam{} doublet.

\begin{figure}
\centering
\includegraphics[width=0.8\textwidth]{figures/linesigma.pdf}
\caption{Distribution of emission-line velocity widths among DEEP2 galaxies at
  $z\sim1$.  The distribution is reasonably well-described by a log-normal
  distribution with mean and standard deviation of
  $\langle\log_{10}\sigma\rangle=1.877\pm0.175$~\kms{} or
  $\langle\sigma\rangle=77$~\kms.  \label{fig:linesigma}}
\end{figure}

\begin{figure}
\centering
\includegraphics[width=0.8\textwidth]{figures/d4000_ewoii.pdf}
\caption{Rest-frame \oii{} equivalent width vs.~the $4000$-\AA{}
  break, $D_{n}(4000)$.  The blue points and uncertainties indicate
  the running median and standard deviation in $0.05$-wide bins of
  $D_{n}(4000)$, while the red curve is a third-order polynomial fit
  to the points with the coefficients given in the
  legend.  The dispersion is relatively uniform with $D_{n}(4000)$
  with a mean value of $0.30$~dex.  {\bf Show the subset of objects
    that satisfy our grz color cuts!}  \label{fig:d4000}}
\end{figure}



\begin{enumerate}
\item{Extension~0: An [{\sc npixel}]$\times$[{\sc ntemplate}] flux
  array which contains {\sc ntemplate} spectra, each {\sc npixel}
  pixels long (see Table~\ref{table:rest} above).}
\item{Extension~1: An [{\sc ntemplate}] data structure with the
  following metadata:
\begin{itemize}
\item{.{\sc TEMPLATEID} - unique template ID number (zero-indexed)}
\end{itemize}
}
\end{enumerate}

The observed-frame template file, {\tt
  elg\_templates\_obs\_VERSION.fits.gz}, is a binary FITS table with
two extensions:

\begin{enumerate}
\item{Extension~0: An [{\sc npixel}]$\times$[{\sc ntemplate}] flux
  array which contains {\sc ntemplate} spectra, each {\sc npixel}
  pixels long (see Table~\ref{table:obs} above).}
\item{Extension~1: An [{\sc ntemplate}] data structure with the
  following metadata:
\begin{itemize}
\item{.{\sc TEMPLATEID} - unique template ID number (zero-indexed)}
\item{.{\sc OBJNO} - unique DEEP2 identifier}
\item{.{\sc RA} - right ascension (degrees, J2000)}
\item{.{\sc DEC} - declination (degrees, J2000)}
\item{.{\sc Z} - heliocentric redshift of the DEEP2 galaxy} 
\item{.{\sc SIGMA\_KMS} - intrinsic velocity line-width (km/s)} 
\item{.{\sc OII\_3726} - \oii~$\lambda3726$ emission-line flux
  (erg~s$^{-1}$~cm$^{^2}$)}  
\item{.{\sc OII\_3729} - \oii~$\lambda3729$ emission-line flux
  (erg~s$^{-1}$~cm$^{^2}$)} 
\item{.{\sc OII\_3727} - total \oiilam{} emission-line flux
  (erg~s$^{-1}$~cm$^{^2}$)} 
\item{.{\sc OII\_3727\_EW} - rest-frame \oii~$\lambda3726,29$
  equivalent width (\AA)}
\item{.{\sc RADIUS\_HALFLIGHT} - half-light radius (arcsec) ($-999$ if not measured)}
\item{.{\sc SERSICN} - Sersic n parameter (concentration) ($-999$ if not measured)}
\item{.{\sc AXIS\_RATIO} - minor-to-major axis ratio ($-999$ if not measured)}
\item{.{\sc DECAM\_G} - synthesized DECam g-band AB magnitude}
\item{.{\sc DECAM\_R} - synthesized DECam r-band AB magnitude}
\item{.{\sc DECAM\_Z} - synthesized DECam z-band AB magnitude}
\item{.{\sc LOGMSTAR} - log10(stellar mass) ($M_{\odot}$, $h=0.7$,
  \citealt{chabrier03a} IMF)} 
\item{.{\sc LOGSFR} - log10(star formation rate)
  ($M_{\odot}$~yr$^{-1}$, $h=0.7$, \citealt{chabrier03a} IMF)}
\item{.{\sc AV\_ISM} - $V$-band continuum attenuation (mag,
  \citealt{charlot00a} attenuation curve)} 
\end{itemize}
}
\end{enumerate}

\subsection{Luminous Red Galaxies (LRGs)}

LRGs


\subsection{Normal Stars (STAR)}

Talk about stuff.

\subsection{F-Type Standard Stars (FSTD)}

\subsection{White Dwarfs (WD)}

Stuff.

\subsection{Bright-Galaxy Survey Galaxies (BGS)}

BGS.  Not yet supported.

\subsection{Quasars (QSO)}

Lyman-alpha quasars.

\subsection{Future Improvements}

Future improvements to these templates may include (in no particular
order): 

\begin{enumerate}
\item{Extending the spectral coverage redward into the mid-infrared so
  that WISE photometry can be synthesized from the models.}
\item{Including additional nebular emission lines.}
\item{Constructing a smaller basis set of models (e.g., archetypes)
  which adequately span the physical parameter space of the full set
  of templates.}
\item{Incorporate more intelligent resampling of the data and models,
  in order to ensure that flux is being conserved correctly.}
\item{Expanding this TechNote to include more quantitative tests of
  the models (e.g., how well they span color-color and color-redshift
  space relative to actual observations).}
\end{enumerate}


\bibliographystyle{apj}
\bibliography{/Users/ioannis/bibdesk/ioannis}
%\input{simulate-templates.bbl}


\clearpage
\appendix

\section{Installation Instructions}\label{sec:install}

\subsection{Basis Templates}

The template-generating code relies on a set of basis templates for each
spectral class.  The latest basis templates are kept at {\tt NERSC} under the
{\tt \$DESI\_ROOT/spectro/templates} top-level directory.  The latest version of
each template file must be downloaded to your local machine, or the code must be
run exclusively at {\tt NERSC}.

Environment variables must be set to point to the latest version of each set of
basis templates.  For example, on {\tt NERSC} (as of the date of this document)
you would define the following environment variables in your {\tt .bashrc.ext}
startup file:
\begin{itemize*}
  \item[\%]{{\tt DESI\_ROOT=/project/projectdirs/desi}}
  \item[\%]{{\tt DESI\_TEMPLATES=\$DESI\_ROOT/spectro/templates}}
  \item[\%]{{\tt DESI\_ELG\_TEMPLATES=\$DESI\_TEMPLATES/elg\_templates/v1.3/elg\_templates\_v1.3.fits}}
  \item[\%]{{\tt DESI\_LRG\_TEMPLATES=\$DESI\_TEMPLATES/lrg\_templates/v1.1/lrg\_templates\_v1.1.fits}}
  \item[\%]{{\tt DESI\_STAR\_TEMPLATES=\$DESI\_TEMPLATES/star\_templates/v1.1/star\_templates\_v1.1.fits}}
  \item[\%]{{\tt DESI\_WD\_TEMPLATES=\$DESI\_TEMPLATES/wd\_templates/v1.0/wd\_templates\_v1.0.fits}}
\end{itemize*}

\subsection{Code}

First, be sure you have recent installations of the {\tt astropy} (needed to
manipulate FITS files), {\tt numpy} (numerical Python library), and {\tt
  matplotlib} (plotting library) packages, or install (or upgrade) them using
{\tt pip}:
\begin{itemize*}
  \item[\%]{{\tt pip install numpy [--upgrade]}}
  \item[\%]{{\tt pip install matplotlib [--upgrade]}}
  \item[\%]{{\tt pip install astropy [--upgrade]}}
\end{itemize*}

\noindent Next, clone the {\tt desihub/desisim} and {\tt desihub/desispec}
repositories.  For example:
\begin{itemize*}
  \item[\%]{{\tt mkdir desihub ; cd desihub}}
  \item[\%]{{\tt git clone https://github.com/desihub/desisim.git}}
  \item[\%]{{\tt git clone https://github.com/desihub/desispec.git}}
\end{itemize*}

\noindent The {\tt desispec} package is needed for the standardized set of I/O
routines which have been written for reading and writing DESI spectra (and the
associated metadata).  Next, add these packages to your {\tt PYTHONPATH}
environment.  For example, in the {\tt bash} shell add the following lines to
your {\tt .bashrc} file:
\begin{itemize*}
  \item[\%]{{\tt DESIHUB=/path/to/desihub}}
  \item[\%]{{\tt
      PYTHONPATH=\$DESIHUB/desisim/py:\$DESIHUB/desispec/py}}
\end{itemize*}

\noindent If you wish, create an alias which points to the top-level executable
script, \simt:
\begin{itemize*}
  \item[\%]{{\tt alias \simt{} \$DESIHUB/desisim/bin/\simt}}
\end{itemize*}

%Random seed can be specified.

To check that the code is installed and in your path try running the following
command, which should print the \emph{help} file for the code.

\begin{verbatim}
% simulate-templates -h
\end{verbatim}

%\begin{verbatim}
%\end{verbatim}

\clearpage
\section{Tables of Nebular Emission Lines}

The following tables summarize the forbidden (Table~\ref{table:forb}) and
hydrogen recombination (Table~\ref{table:recomb}) lines included in the
emission-line spectrum class.  Additional lines may be added in the future. 

%\begin{center}
%\begin{table}[h!]
%\begin{tabular}{cc}
%\hline
%\hline
%Emission Line & Vacuum Wavelength (\AA) \\
%\hline
% \oii~$\lambda3726$ & 3727.092 \\
% \oii~$\lambda3729$ & 3729.874 \\
%\oiii~$\lambda5007$ & 5008.239 \\
%\oiii~$\lambda4959$ & 4960.295 \\
% \nii~$\lambda6584$ & 6585.277 \\
% \nii~$\lambda6548$ & 6549.852 \\
% \sii~$\lambda6716$ & 6718.294 \\
% \sii~$\lambda6731$ & 6732.673 \\
%\hline
%\end{tabular}
%\caption{Stuff}
%\end{table}
%\end{center}

\begin{deluxetable}{cc}
\tablecaption{Nebular forbidden lines\label{table:forb}}
\tablewidth{0pt}
\tablehead{\colhead{Emission Line} & \colhead{Vacuum Wavelength (\AA)}}
\startdata
 \oii~$\lambda3726$ & 3727.092 \\
 \oii~$\lambda3729$ & 3729.874 \\
\oiii~$\lambda5007$ & 5008.239 \\
\oiii~$\lambda4959$ & 4960.295 \\
 \nii~$\lambda6584$ & 6585.277 \\
 \nii~$\lambda6548$ & 6549.852 \\
 \sii~$\lambda6716$ & 6718.294 \\
 \sii~$\lambda6731$ & 6732.673
\enddata
\end{deluxetable}


\begin{deluxetable}{ccc}
\tablecaption{Nebular recombination lines\label{table:recomb}} 
\tablewidth{0pt}
\tablehead{\colhead{Emission Line} & \colhead{Vacuum Wavelength (\AA)} &
  \colhead{Ratio\tablenotemark{a}}} 
\startdata
Ly$\alpha$ & 1215.670 &    23.54656 \\
\cline{1-3}
H$\alpha$ & 6564.613 &    2.863158 \\
    H$\beta$ & 4862.683 &    1.000000 \\
   H$\gamma$ & 4341.684 &   0.4683401 \\
   H$\delta$ & 4102.892 &   0.2589474 \\
 H$\epsilon$ & 3971.195 &   0.1590283 \\
       H8 & 3890.151 &   0.1050202 \\
       H9 & 3836.472 &  0.07312550 \\
      H10 & 3798.976 &  0.05302830 \\
      H11 & 3771.701 &  0.03971660 \\
      H12 & 3751.217 &  0.03053440 \\
      H13 & 3735.430 &  0.02399190 \\
      H14 & 3722.997 &  0.01920650 \\
      H15 & 3713.027 &  0.01561130 \\
      H16 & 3704.906 &  0.01286640 \\
      H17 & 3698.204 &  0.01073680 \\
      H18 & 3692.605 & 0.009052600 \\
\cline{1-3}
  Pa$\alpha$ & 18756.13 &   0.3386235 \\
   Pa$\beta$ & 12821.59 &   0.1632389 \\
  Pa$\gamma$ & 10941.09 &  0.09044530 \\
  Pa$\delta$ & 10052.13 &  0.05545750 \\
Pa$\epsilon$ & 9548.591 &  0.03656680 \\
      Pa9 & 9231.547 &  0.02543320 \\
     Pa10 & 9017.385 &  0.01842910 \\
     Pa11 & 8865.217 &  0.01378950 \\
      P12 & 8752.876 &  0.01059920 \\
\cline{1-3}
  Br$\alpha$ & 40522.62 &  0.08021050 \\
   Br$\beta$ & 26258.67 &  0.04546560 \\
  Br$\gamma$ & 21661.20 &  0.02781380 \\
  Br$\delta$ & 19450.87 &  0.01825910 \\
\cline{1-3}
\hei~$\lambda3188$ & 3188.662 &  0.04076080 \\
\hei~$\lambda3389$ & 3889.752 &   0.1027011 \\
\hei~$\lambda3965$ & 3965.852 &  0.01022650 \\
\hei~$\lambda4026$ & 4027.348 &  0.02115760 \\
\hei~$\lambda4471$ & 4472.755 &  0.04445050 \\
\hei~$\lambda4922$ & 4923.304 &  0.01199870 \\
\hei~$\lambda5016$ & 5017.079 &  0.02568980 \\
\hei~$\lambda5876$ & 5877.299 &   0.1186800 \\
\hei~$\lambda6678$ & 6679.994 &  0.03360660 \\
\hei~$\lambda7065$ & 7067.198 &  0.02173860 \\
\hei~$\lambda10830$ & 10833.22 &   0.2404832
\enddata
\tablenotetext{a}{Ratio with respect to H$\beta$ assuming case~B recombination.} 
\end{deluxetable}

\end{document}





\clearpage
\appendix

\section{Installation Instructions}\label{sec:install}

\subsection{Basis Templates}

The template-generating code relies on a set of basis templates for each
spectral class.  The latest basis templates are kept at {\tt NERSC} under the
{\tt \$DESI\_ROOT/spectro/templates} top-level directory.  The latest version of
each template file must be downloaded to your local machine, or the code must be
run exclusively at {\tt NERSC}.

Environment variables must be set to point to the latest version of each set of
basis templates.  For example, on {\tt NERSC} (as of the date of this document)
you would define the following environment variables in your {\tt .bashrc.ext}
startup file:
\begin{itemize*}
  \item[\%]{{\tt DESI\_ROOT=/project/projectdirs/desi}}
  \item[\%]{{\tt DESI\_TEMPLATES=\$DESI\_ROOT/spectro/templates}}
  \item[\%]{{\tt DESI\_ELG\_TEMPLATES=\$DESI\_TEMPLATES/elg\_templates/v1.3/elg\_templates\_v1.3.fits}}
  \item[\%]{{\tt DESI\_LRG\_TEMPLATES=\$DESI\_TEMPLATES/lrg\_templates/v1.1/lrg\_templates\_v1.1.fits}}
  \item[\%]{{\tt DESI\_STAR\_TEMPLATES=\$DESI\_TEMPLATES/star\_templates/v1.1/star\_templates\_v1.1.fits}}
  \item[\%]{{\tt DESI\_WD\_TEMPLATES=\$DESI\_TEMPLATES/wd\_templates/v1.0/wd\_templates\_v1.0.fits}}
\end{itemize*}

\subsection{Code}

First, be sure you have recent installations of the {\tt astropy} (needed to
manipulate FITS files), {\tt numpy} (numerical Python library), and {\tt
  matplotlib} (plotting library) packages, or install (or upgrade) them using
{\tt pip}:
\begin{itemize*}
  \item[\%]{{\tt pip install numpy [--upgrade]}}
  \item[\%]{{\tt pip install matplotlib [--upgrade]}}
  \item[\%]{{\tt pip install astropy [--upgrade]}}
\end{itemize*}

\noindent Next, clone the {\tt desihub/desisim} and {\tt desihub/desispec}
repositories.  For example:
\begin{itemize*}
  \item[\%]{{\tt mkdir desihub ; cd desihub}}
  \item[\%]{{\tt git clone https://github.com/desihub/desisim.git}}
  \item[\%]{{\tt git clone https://github.com/desihub/desispec.git}}
\end{itemize*}

\noindent The {\tt desispec} package is needed for the standardized set of I/O
routines which have been written for reading and writing DESI spectra (and the
associated metadata).  Next, add these packages to your {\tt PYTHONPATH}
environment.  For example, in the {\tt bash} shell add the following lines to
your {\tt .bashrc} file:
\begin{itemize*}
  \item[\%]{{\tt DESIHUB=/path/to/desihub}}
  \item[\%]{{\tt
      PYTHONPATH=\$DESIHUB/desisim/py:\$DESIHUB/desispec/py}}
\end{itemize*}

\noindent If you wish, create an alias which points to the top-level executable
script, \simt:
\begin{itemize*}
  \item[\%]{{\tt alias \simt{} \$DESIHUB/desisim/bin/\simt}}
\end{itemize*}

%Random seed can be specified.

To check that the code is installed and in your path try running the following
command, which should print the \emph{help} file for the code.

\begin{verbatim}
% simulate-templates -h
\end{verbatim}

%\begin{verbatim}
%\end{verbatim}

\clearpage
\section{Tables of Nebular Emission Lines}

The following tables summarize the forbidden (Table~\ref{table:forb}) and
hydrogen recombination (Table~\ref{table:recomb}) lines included in the
emission-line spectrum class.  Additional lines may be added in the future. 

%\begin{center}
%\begin{table}[h!]
%\begin{tabular}{cc}
%\hline
%\hline
%Emission Line & Vacuum Wavelength (\AA) \\
%\hline
% \oii~$\lambda3726$ & 3727.092 \\
% \oii~$\lambda3729$ & 3729.874 \\
%\oiii~$\lambda5007$ & 5008.239 \\
%\oiii~$\lambda4959$ & 4960.295 \\
% \nii~$\lambda6584$ & 6585.277 \\
% \nii~$\lambda6548$ & 6549.852 \\
% \sii~$\lambda6716$ & 6718.294 \\
% \sii~$\lambda6731$ & 6732.673 \\
%\hline
%\end{tabular}
%\caption{Stuff}
%\end{table}
%\end{center}

\begin{deluxetable}{cc}
\tablecaption{Nebular forbidden lines\label{table:forb}}
\tablewidth{0pt}
\tablehead{\colhead{Emission Line} & \colhead{Vacuum Wavelength (\AA)}}
\startdata
 \oii~$\lambda3726$ & 3727.092 \\
 \oii~$\lambda3729$ & 3729.874 \\
\oiii~$\lambda5007$ & 5008.239 \\
\oiii~$\lambda4959$ & 4960.295 \\
 \nii~$\lambda6584$ & 6585.277 \\
 \nii~$\lambda6548$ & 6549.852 \\
 \sii~$\lambda6716$ & 6718.294 \\
 \sii~$\lambda6731$ & 6732.673
\enddata
\end{deluxetable}


\begin{deluxetable}{ccc}
\tablecaption{Nebular recombination lines\label{table:recomb}} 
\tablewidth{0pt}
\tablehead{\colhead{Emission Line} & \colhead{Vacuum Wavelength (\AA)} &
  \colhead{Ratio\tablenotemark{a}}} 
\startdata
Ly$\alpha$ & 1215.670 &    23.54656 \\
\cline{1-3}
H$\alpha$ & 6564.613 &    2.863158 \\
    H$\beta$ & 4862.683 &    1.000000 \\
   H$\gamma$ & 4341.684 &   0.4683401 \\
   H$\delta$ & 4102.892 &   0.2589474 \\
 H$\epsilon$ & 3971.195 &   0.1590283 \\
       H8 & 3890.151 &   0.1050202 \\
       H9 & 3836.472 &  0.07312550 \\
      H10 & 3798.976 &  0.05302830 \\
      H11 & 3771.701 &  0.03971660 \\
      H12 & 3751.217 &  0.03053440 \\
      H13 & 3735.430 &  0.02399190 \\
      H14 & 3722.997 &  0.01920650 \\
      H15 & 3713.027 &  0.01561130 \\
      H16 & 3704.906 &  0.01286640 \\
      H17 & 3698.204 &  0.01073680 \\
      H18 & 3692.605 & 0.009052600 \\
\cline{1-3}
  Pa$\alpha$ & 18756.13 &   0.3386235 \\
   Pa$\beta$ & 12821.59 &   0.1632389 \\
  Pa$\gamma$ & 10941.09 &  0.09044530 \\
  Pa$\delta$ & 10052.13 &  0.05545750 \\
Pa$\epsilon$ & 9548.591 &  0.03656680 \\
      Pa9 & 9231.547 &  0.02543320 \\
     Pa10 & 9017.385 &  0.01842910 \\
     Pa11 & 8865.217 &  0.01378950 \\
      P12 & 8752.876 &  0.01059920 \\
\cline{1-3}
  Br$\alpha$ & 40522.62 &  0.08021050 \\
   Br$\beta$ & 26258.67 &  0.04546560 \\
  Br$\gamma$ & 21661.20 &  0.02781380 \\
  Br$\delta$ & 19450.87 &  0.01825910 \\
\cline{1-3}
\hei~$\lambda3188$ & 3188.662 &  0.04076080 \\
\hei~$\lambda3389$ & 3889.752 &   0.1027011 \\
\hei~$\lambda3965$ & 3965.852 &  0.01022650 \\
\hei~$\lambda4026$ & 4027.348 &  0.02115760 \\
\hei~$\lambda4471$ & 4472.755 &  0.04445050 \\
\hei~$\lambda4922$ & 4923.304 &  0.01199870 \\
\hei~$\lambda5016$ & 5017.079 &  0.02568980 \\
\hei~$\lambda5876$ & 5877.299 &   0.1186800 \\
\hei~$\lambda6678$ & 6679.994 &  0.03360660 \\
\hei~$\lambda7065$ & 7067.198 &  0.02173860 \\
\hei~$\lambda10830$ & 10833.22 &   0.2404832
\enddata
\tablenotetext{a}{Ratio with respect to H$\beta$ assuming case~B recombination.} 
\end{deluxetable}

\end{document}





\clearpage
\appendix

\section{Installation Instructions}\label{sec:install}

\subsection{Basis Templates}

The template-generating code relies on a set of basis templates for each
spectral class.  The latest basis templates are kept at {\tt NERSC} under the
{\tt \$DESI\_ROOT/spectro/templates} top-level directory.  The latest version of
each template file must be downloaded to your local machine, or the code must be
run exclusively at {\tt NERSC}.

Environment variables must be set to point to the latest version of each set of
basis templates.  For example, on {\tt NERSC} (as of the date of this document)
you would define the following environment variables in your {\tt .bashrc.ext}
startup file:
\begin{itemize*}
  \item[\%]{{\tt DESI\_ROOT=/project/projectdirs/desi}}
  \item[\%]{{\tt DESI\_TEMPLATES=\$DESI\_ROOT/spectro/templates}}
  \item[\%]{{\tt DESI\_ELG\_TEMPLATES=\$DESI\_TEMPLATES/elg\_templates/v1.3/elg\_templates\_v1.3.fits}}
  \item[\%]{{\tt DESI\_LRG\_TEMPLATES=\$DESI\_TEMPLATES/lrg\_templates/v1.1/lrg\_templates\_v1.1.fits}}
  \item[\%]{{\tt DESI\_STAR\_TEMPLATES=\$DESI\_TEMPLATES/star\_templates/v1.1/star\_templates\_v1.1.fits}}
  \item[\%]{{\tt DESI\_WD\_TEMPLATES=\$DESI\_TEMPLATES/wd\_templates/v1.0/wd\_templates\_v1.0.fits}}
\end{itemize*}

\subsection{Code}

First, be sure you have recent installations of the {\tt astropy} (needed to
manipulate FITS files), {\tt numpy} (numerical Python library), and {\tt
  matplotlib} (plotting library) packages, or install (or upgrade) them using
{\tt pip}:
\begin{itemize*}
  \item[\%]{{\tt pip install numpy [--upgrade]}}
  \item[\%]{{\tt pip install matplotlib [--upgrade]}}
  \item[\%]{{\tt pip install astropy [--upgrade]}}
\end{itemize*}

\noindent Next, clone the {\tt desihub/desisim} and {\tt desihub/desispec}
repositories.  For example:
\begin{itemize*}
  \item[\%]{{\tt mkdir desihub ; cd desihub}}
  \item[\%]{{\tt git clone https://github.com/desihub/desisim.git}}
  \item[\%]{{\tt git clone https://github.com/desihub/desispec.git}}
\end{itemize*}

\noindent The {\tt desispec} package is needed for the standardized set of I/O
routines which have been written for reading and writing DESI spectra (and the
associated metadata).  Next, add these packages to your {\tt PYTHONPATH}
environment.  For example, in the {\tt bash} shell add the following lines to
your {\tt .bashrc} file:
\begin{itemize*}
  \item[\%]{{\tt DESIHUB=/path/to/desihub}}
  \item[\%]{{\tt
      PYTHONPATH=\$DESIHUB/desisim/py:\$DESIHUB/desispec/py}}
\end{itemize*}

\noindent If you wish, create an alias which points to the top-level executable
script, \simt:
\begin{itemize*}
  \item[\%]{{\tt alias \simt{} \$DESIHUB/desisim/bin/\simt}}
\end{itemize*}

%Random seed can be specified.

To check that the code is installed and in your path try running the following
command, which should print the \emph{help} file for the code.

\begin{verbatim}
% simulate-templates -h
\end{verbatim}

%\begin{verbatim}
%\end{verbatim}

\clearpage
\section{Tables of Nebular Emission Lines}

The following tables summarize the forbidden (Table~\ref{table:forb}) and
hydrogen recombination (Table~\ref{table:recomb}) lines included in the
emission-line spectrum class.  Additional lines may be added in the future. 

%\begin{center}
%\begin{table}[h!]
%\begin{tabular}{cc}
%\hline
%\hline
%Emission Line & Vacuum Wavelength (\AA) \\
%\hline
% \oii~$\lambda3726$ & 3727.092 \\
% \oii~$\lambda3729$ & 3729.874 \\
%\oiii~$\lambda5007$ & 5008.239 \\
%\oiii~$\lambda4959$ & 4960.295 \\
% \nii~$\lambda6584$ & 6585.277 \\
% \nii~$\lambda6548$ & 6549.852 \\
% \sii~$\lambda6716$ & 6718.294 \\
% \sii~$\lambda6731$ & 6732.673 \\
%\hline
%\end{tabular}
%\caption{Stuff}
%\end{table}
%\end{center}

\begin{deluxetable}{cc}
\tablecaption{Nebular forbidden lines\label{table:forb}}
\tablewidth{0pt}
\tablehead{\colhead{Emission Line} & \colhead{Vacuum Wavelength (\AA)}}
\startdata
 \oii~$\lambda3726$ & 3727.092 \\
 \oii~$\lambda3729$ & 3729.874 \\
\oiii~$\lambda5007$ & 5008.239 \\
\oiii~$\lambda4959$ & 4960.295 \\
 \nii~$\lambda6584$ & 6585.277 \\
 \nii~$\lambda6548$ & 6549.852 \\
 \sii~$\lambda6716$ & 6718.294 \\
 \sii~$\lambda6731$ & 6732.673
\enddata
\end{deluxetable}


\begin{deluxetable}{ccc}
\tablecaption{Nebular recombination lines\label{table:recomb}} 
\tablewidth{0pt}
\tablehead{\colhead{Emission Line} & \colhead{Vacuum Wavelength (\AA)} &
  \colhead{Ratio\tablenotemark{a}}} 
\startdata
Ly$\alpha$ & 1215.670 &    23.54656 \\
\cline{1-3}
H$\alpha$ & 6564.613 &    2.863158 \\
    H$\beta$ & 4862.683 &    1.000000 \\
   H$\gamma$ & 4341.684 &   0.4683401 \\
   H$\delta$ & 4102.892 &   0.2589474 \\
 H$\epsilon$ & 3971.195 &   0.1590283 \\
       H8 & 3890.151 &   0.1050202 \\
       H9 & 3836.472 &  0.07312550 \\
      H10 & 3798.976 &  0.05302830 \\
      H11 & 3771.701 &  0.03971660 \\
      H12 & 3751.217 &  0.03053440 \\
      H13 & 3735.430 &  0.02399190 \\
      H14 & 3722.997 &  0.01920650 \\
      H15 & 3713.027 &  0.01561130 \\
      H16 & 3704.906 &  0.01286640 \\
      H17 & 3698.204 &  0.01073680 \\
      H18 & 3692.605 & 0.009052600 \\
\cline{1-3}
  Pa$\alpha$ & 18756.13 &   0.3386235 \\
   Pa$\beta$ & 12821.59 &   0.1632389 \\
  Pa$\gamma$ & 10941.09 &  0.09044530 \\
  Pa$\delta$ & 10052.13 &  0.05545750 \\
Pa$\epsilon$ & 9548.591 &  0.03656680 \\
      Pa9 & 9231.547 &  0.02543320 \\
     Pa10 & 9017.385 &  0.01842910 \\
     Pa11 & 8865.217 &  0.01378950 \\
      P12 & 8752.876 &  0.01059920 \\
\cline{1-3}
  Br$\alpha$ & 40522.62 &  0.08021050 \\
   Br$\beta$ & 26258.67 &  0.04546560 \\
  Br$\gamma$ & 21661.20 &  0.02781380 \\
  Br$\delta$ & 19450.87 &  0.01825910 \\
\cline{1-3}
\hei~$\lambda3188$ & 3188.662 &  0.04076080 \\
\hei~$\lambda3389$ & 3889.752 &   0.1027011 \\
\hei~$\lambda3965$ & 3965.852 &  0.01022650 \\
\hei~$\lambda4026$ & 4027.348 &  0.02115760 \\
\hei~$\lambda4471$ & 4472.755 &  0.04445050 \\
\hei~$\lambda4922$ & 4923.304 &  0.01199870 \\
\hei~$\lambda5016$ & 5017.079 &  0.02568980 \\
\hei~$\lambda5876$ & 5877.299 &   0.1186800 \\
\hei~$\lambda6678$ & 6679.994 &  0.03360660 \\
\hei~$\lambda7065$ & 7067.198 &  0.02173860 \\
\hei~$\lambda10830$ & 10833.22 &   0.2404832
\enddata
\tablenotetext{a}{Ratio with respect to H$\beta$ assuming case~B recombination.} 
\end{deluxetable}

\end{document}





\clearpage
\appendix

\section{Installation Instructions}\label{sec:install}

\subsection{Basis Templates}

The template-generating code relies on a set of basis templates for each
spectral class.  The latest basis templates are kept at {\tt NERSC} under the
{\tt \$DESI\_ROOT/spectro/templates} top-level directory.  The latest version of
each template file must be downloaded to your local machine, or the code must be
run exclusively at {\tt NERSC}.

Environment variables must be set to point to the latest version of each set of
basis templates.  For example, on {\tt NERSC} (as of the date of this document)
you would define the following environment variables in your {\tt .bashrc.ext}
startup file:
\begin{itemize*}
  \item[\%]{{\tt DESI\_ROOT=/project/projectdirs/desi}}
  \item[\%]{{\tt DESI\_TEMPLATES=\$DESI\_ROOT/spectro/templates}}
  \item[\%]{{\tt DESI\_ELG\_TEMPLATES=\$DESI\_TEMPLATES/elg\_templates/v1.3/elg\_templates\_v1.3.fits}}
  \item[\%]{{\tt DESI\_LRG\_TEMPLATES=\$DESI\_TEMPLATES/lrg\_templates/v1.1/lrg\_templates\_v1.1.fits}}
  \item[\%]{{\tt DESI\_STAR\_TEMPLATES=\$DESI\_TEMPLATES/star\_templates/v1.1/star\_templates\_v1.1.fits}}
  \item[\%]{{\tt DESI\_WD\_TEMPLATES=\$DESI\_TEMPLATES/wd\_templates/v1.0/wd\_templates\_v1.0.fits}}
\end{itemize*}

\subsection{Code}

First, be sure you have recent installations of the {\tt astropy} (needed to
manipulate FITS files), {\tt numpy} (numerical Python library), and {\tt
  matplotlib} (plotting library) packages, or install (or upgrade) them using
{\tt pip}:
\begin{itemize*}
  \item[\%]{{\tt pip install numpy [--upgrade]}}
  \item[\%]{{\tt pip install matplotlib [--upgrade]}}
  \item[\%]{{\tt pip install astropy [--upgrade]}}
\end{itemize*}

\noindent Next, clone the {\tt desihub/desisim} and {\tt desihub/desispec}
repositories.  For example:
\begin{itemize*}
  \item[\%]{{\tt mkdir desihub ; cd desihub}}
  \item[\%]{{\tt git clone https://github.com/desihub/desisim.git}}
  \item[\%]{{\tt git clone https://github.com/desihub/desispec.git}}
\end{itemize*}

\noindent The {\tt desispec} package is needed for the standardized set of I/O
routines which have been written for reading and writing DESI spectra (and the
associated metadata).  Next, add these packages to your {\tt PYTHONPATH}
environment.  For example, in the {\tt bash} shell add the following lines to
your {\tt .bashrc} file:
\begin{itemize*}
  \item[\%]{{\tt DESIHUB=/path/to/desihub}}
  \item[\%]{{\tt
      PYTHONPATH=\$DESIHUB/desisim/py:\$DESIHUB/desispec/py}}
\end{itemize*}

\noindent If you wish, create an alias which points to the top-level executable
script, \simt:
\begin{itemize*}
  \item[\%]{{\tt alias \simt{} \$DESIHUB/desisim/bin/\simt}}
\end{itemize*}

%Random seed can be specified.

To check that the code is installed and in your path try running the following
command, which should print the \emph{help} file for the code.

\begin{verbatim}
% simulate-templates -h
\end{verbatim}

%\begin{verbatim}
%\end{verbatim}

\clearpage
\section{Tables of Nebular Emission Lines}

The following tables summarize the forbidden (Table~\ref{table:forb}) and
hydrogen recombination (Table~\ref{table:recomb}) lines included in the
emission-line spectrum class.  Additional lines may be added in the future. 

%\begin{center}
%\begin{table}[h!]
%\begin{tabular}{cc}
%\hline
%\hline
%Emission Line & Vacuum Wavelength (\AA) \\
%\hline
% \oii~$\lambda3726$ & 3727.092 \\
% \oii~$\lambda3729$ & 3729.874 \\
%\oiii~$\lambda5007$ & 5008.239 \\
%\oiii~$\lambda4959$ & 4960.295 \\
% \nii~$\lambda6584$ & 6585.277 \\
% \nii~$\lambda6548$ & 6549.852 \\
% \sii~$\lambda6716$ & 6718.294 \\
% \sii~$\lambda6731$ & 6732.673 \\
%\hline
%\end{tabular}
%\caption{Stuff}
%\end{table}
%\end{center}

\begin{deluxetable}{cc}
\tablecaption{Nebular forbidden lines\label{table:forb}}
\tablewidth{0pt}
\tablehead{\colhead{Emission Line} & \colhead{Vacuum Wavelength (\AA)}}
\startdata
 \oii~$\lambda3726$ & 3727.092 \\
 \oii~$\lambda3729$ & 3729.874 \\
\oiii~$\lambda5007$ & 5008.239 \\
\oiii~$\lambda4959$ & 4960.295 \\
 \nii~$\lambda6584$ & 6585.277 \\
 \nii~$\lambda6548$ & 6549.852 \\
 \sii~$\lambda6716$ & 6718.294 \\
 \sii~$\lambda6731$ & 6732.673
\enddata
\end{deluxetable}


\begin{deluxetable}{ccc}
\tablecaption{Nebular recombination lines\label{table:recomb}} 
\tablewidth{0pt}
\tablehead{\colhead{Emission Line} & \colhead{Vacuum Wavelength (\AA)} &
  \colhead{Ratio\tablenotemark{a}}} 
\startdata
Ly$\alpha$ & 1215.670 &    23.54656 \\
\cline{1-3}
H$\alpha$ & 6564.613 &    2.863158 \\
    H$\beta$ & 4862.683 &    1.000000 \\
   H$\gamma$ & 4341.684 &   0.4683401 \\
   H$\delta$ & 4102.892 &   0.2589474 \\
 H$\epsilon$ & 3971.195 &   0.1590283 \\
       H8 & 3890.151 &   0.1050202 \\
       H9 & 3836.472 &  0.07312550 \\
      H10 & 3798.976 &  0.05302830 \\
      H11 & 3771.701 &  0.03971660 \\
      H12 & 3751.217 &  0.03053440 \\
      H13 & 3735.430 &  0.02399190 \\
      H14 & 3722.997 &  0.01920650 \\
      H15 & 3713.027 &  0.01561130 \\
      H16 & 3704.906 &  0.01286640 \\
      H17 & 3698.204 &  0.01073680 \\
      H18 & 3692.605 & 0.009052600 \\
\cline{1-3}
  Pa$\alpha$ & 18756.13 &   0.3386235 \\
   Pa$\beta$ & 12821.59 &   0.1632389 \\
  Pa$\gamma$ & 10941.09 &  0.09044530 \\
  Pa$\delta$ & 10052.13 &  0.05545750 \\
Pa$\epsilon$ & 9548.591 &  0.03656680 \\
      Pa9 & 9231.547 &  0.02543320 \\
     Pa10 & 9017.385 &  0.01842910 \\
     Pa11 & 8865.217 &  0.01378950 \\
      P12 & 8752.876 &  0.01059920 \\
\cline{1-3}
  Br$\alpha$ & 40522.62 &  0.08021050 \\
   Br$\beta$ & 26258.67 &  0.04546560 \\
  Br$\gamma$ & 21661.20 &  0.02781380 \\
  Br$\delta$ & 19450.87 &  0.01825910 \\
\cline{1-3}
\hei~$\lambda3188$ & 3188.662 &  0.04076080 \\
\hei~$\lambda3389$ & 3889.752 &   0.1027011 \\
\hei~$\lambda3965$ & 3965.852 &  0.01022650 \\
\hei~$\lambda4026$ & 4027.348 &  0.02115760 \\
\hei~$\lambda4471$ & 4472.755 &  0.04445050 \\
\hei~$\lambda4922$ & 4923.304 &  0.01199870 \\
\hei~$\lambda5016$ & 5017.079 &  0.02568980 \\
\hei~$\lambda5876$ & 5877.299 &   0.1186800 \\
\hei~$\lambda6678$ & 6679.994 &  0.03360660 \\
\hei~$\lambda7065$ & 7067.198 &  0.02173860 \\
\hei~$\lambda10830$ & 10833.22 &   0.2404832
\enddata
\tablenotetext{a}{Ratio with respect to H$\beta$ assuming case~B recombination.} 
\end{deluxetable}

\end{document}


